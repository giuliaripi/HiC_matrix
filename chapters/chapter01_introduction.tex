The Hi-C method is a method which enables the comprehensive, unbiased mapping of long-range chromatin interactions across an entire genome. Hi-C couples proximity-based ligation with massively parallel sequencing to probe the 3D architecture of whole genomes.
The underlying principle of Hi-C involves crosslinking cells with formaldehyde, digesting the DNA, filling the overhangs with a biotinylated residue, and then performing ligation under dilute conditions to favor events between cross-linked fragments. 
The resulting biotin-marked ligation products, which represent fragments originally in close spatial proximity, are then purified, sheared, and analyzed using massively parallel DNA sequencing. 
This process allows for the construction of a genome-wide contact matrix M at a resolution of 1 megabase, reflecting the ensemble average of interactions (\cite{comprehensive_mapping}).
\subsection{Hi-C matrix}
From the Hi-C experimental measure that describes how DNA is folded in three-dimensional space within the cell nucleus, the end result is a Hi-C matrix, which represents the probability that two genomic regions are spatially close. 
The matrix is essentially a two-dimensional heatmap where both axes represent genomic coordinates. 
It is square and symmetrical: each element $(i, j)$ indicates how many times the regions $i$ and $j$ are in contact.
The diagonal of the matrix represents interactions of genomic regions with themselves or nearby regions while off-diagonal elements indicate long-range interactions, including loops and contacts between different chromosomal domains.
Since the experimental data vary in reading depth, they are normalized, so they do not contain integer values but normalized real values.
From a biological point of view, HI-C matrices are used to identify Topologically Associating Domains (TADs), which are regions with enriched internal interactions, chromatin loops, which are point interactions between enhancers and promoters and A/B compartments, which are large-scale separation of active (A) and inactive (B) chromatin. 
