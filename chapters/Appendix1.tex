In the main text, the eigenvalue spectrum is shown for individual chromosomes. While this chromosome-specific analysis is required for signal–noise separation and for the generation of synthetic Hi-C matrices, it is known that Random Matrix Theory predictions, such as the Wigner semicircle law, hold asymptotically and are best observed when averaging over large matrix ensembles.
To verify the consistency of the noise component with Random Matrix Theory at a global level, we therefore aggregated the eigenvalues obtained from all chromosomes considered in this study. Figure \ref{} shows the resulting eigenvalue distribution, overlaid with the corresponding Wigner semicircle.
Compared to single-chromosome spectra, the aggregated distribution exhibits a much closer agreement with the semicircular law. This confirms that deviations observed at the individual chromosome level are mainly due to finite-size effects and chromosome-specific structural correlations, rather than to a breakdown of the Random Matrix Theory assumptions.
This analysis serves as a global statistical validation of the RMT-based framework, while all subsequent signal–noise decomposition and synthetic data generation steps are performed independently for each chromosome in order to preserve their specific biological and structural properties.


%Didascalia
%Eigenvalue distribution obtained by aggregating all chromosomes analyzed in this study. The histogram is overlaid with the Wigner semicircle predicted by Random Matrix Theory. The improved agreement compared to single-chromosome spectra highlights the role of finite-size and chromosome-specific effects.
