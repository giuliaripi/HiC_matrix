\begin{titlepage}
    \begin{center}
        \thispagestyle{empty}

        {\LARGE \textbf{Synthetic HI-C matrix generation} \par}  


        \vspace{1.0cm}
        
        {\large Giulia Ripi}  

        \vspace{0.3cm}
        
        {\large 11/01/2026}
        
        \vfill
        \newpage
        \textbf{Abstract}
    \end{center}

    \noindent
    
    Hi-C experiments generate genome-wide contact matrices that encode the three-dimensional organization of chromatin, but their analysis is often affected by experimental noise and limited data availability. In this work, we present a framework for the generation of realistic synthetic Hi-C matrices that preserve the structural properties of experimental data. Using Random Matrix Theory, Hi-C matrices are spectrally decomposed to separate biologically meaningful signal from random noise, exploiting the Wigner semicircle law to identify noise-dominated eigenmodes. Synthetic variability is introduced by perturbing the noise component at the eigenvector level while preserving the original eigenvalue spectrum, and synthetic matrices are reconstructed by recombining signal and structured noise.
    To validate the realism of the synthetic data, Hi-C matrices are modeled as weighted networks and compared to real data using multiple centrality measures. Statistical agreement is assessed through z-scores, empirical p-values with false discovery rate correction, and Kolmogorov–Smirnov tests, as well as clustering and embedding analyses. The results show that the synthetic matrices are statistically indistinguishable from real data, supporting the effectiveness of the proposed approach.

    \vfill
\end{titlepage}